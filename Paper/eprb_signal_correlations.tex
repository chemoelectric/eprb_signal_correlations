\documentclass[9pt,technote]{IEEEtran}
%\documentclass[12pt,journal,onecolumn]{IEEEtran}

\usepackage{amsmath,amssymb,amsfonts}
\usepackage{algorithmic}
\usepackage{algorithm}
\usepackage{array}
\usepackage[caption=false,font=normalsize,labelfont=sf,textfont=sf]{subfig}
\usepackage{textcomp}
\usepackage{stfloats}
\usepackage{url}
\usepackage{verbatim}
\usepackage{graphicx}
\usepackage{hyperref}
\usepackage{cite}
\hyphenation{op-tical net-works semi-conduc-tor IEEE-Xplore}

\begin{document}

\title{Correlation Coefficient of a Two-Channel Bell Test\\by
  Probability Theory}
\author{Barry~Schwartz,~\IEEEmembership{Member,~IEEE}}

\maketitle
\insert\footins{%
  % See
  % https://tex.stackexchange.com/questions/117597/inserting-a-creative-commons-licence-into-a-latex-document
  \normalfont\footnotesize\raggedright
  \interlinepenalty\interfootnotelinepenalty \splittopskip\footnotesep
  \splitmaxdepth \dp\strutbox \floatingpenalty10000 \hsize\columnwidth
  Copyright \copyright\ 2023 Barry Schwartz \par
  Revision: \input{eprb_signal_correlations_revision.txt}\par
  This work is licensed under the Creative Commons Attribution~4.0
  International License. To view a copy of this license, visit
  \href{http://creativecommons.org/licenses/by/4.0/}{http://creativecommons.org/licenses/by/4.0/}
  or send a letter to Creative Commons, P.~O. Box~1866, Mountain~View,
  CA~94042, USA.\par
  A PDF copy of this paper may be found at
  \href{https://www.crudfactory.com/eprb_signal_correlations.pdf}{https://www.crudfactory.com/eprb\_signal\_correlations.pdf}\par
  The \LaTeX\ source for this paper, and simulations based on its
  contents, reside in a Git repository at
  \href{https://github.com/chemoelectric/eprb_signal_correlations}{https://github.com/chemoelectric/eprb\_signal\_correlations}\par}

\begin{abstract}
  It is widely believed that only quantum mechanics can characterize
  the correlations of a Bell test. In fact, quantum mechanics is
  superfluous to the problem, and a correlation coefficient may be
  derived by probability theory. There is no ``entanglement.''
  Everything in the experiment takes place by contact.
\end{abstract}

\section{Introduction}

John~S. Bell famously proposed a test supposed to prove that quantum
mechanics was, in essence, magical~\cite{enwiki:1174875317}. I put the
matter this way because the outcome of such an experiment is soluble by
probability theory, and yet Bell seems to have claimed quantum
mechanics alone could provide the One True Answer that surpassed all
other solutions. At the time of this writing, Bell's test is
considered passed and the matter resolved in favor of the One True
Answer.

Of course, what actually happened is that Bell {\em{did not know}} the
test could be analyzed by means other than quantum mechanics, and also
surely {\em{already entertained the notion}} that quantum mechanics
might indeed be magical. Quantum mechanics has widely been regarded as
some kind of ``magic'' for a very long time. However, some of us do
not---at least when supposedly engaged in the scientific
endeavor---entertain the notion that magic---or, indeed, anything
remotely resembling it---exists. Furthermore, the Bell test problem is
not, in fact, one of quantum mechanics at all. It is more properly a
problem in random process analysis. It merely happens that quantum
mechanics ``knows'' how to solve such a problem, without the ``user''
of quantum mechanics having to understand one iota of the subject of
random process analysis.

You and I, however, wish to understand---and so shall we recast the
problem as one of random signal analysis. We shall state a problem in
random signal analysis, and the reader may confirm for themself that
it is equivalent to a two-channel Bell test as described
in~\cite{enwiki:1174875317}. Then we shall derive a correlation
coefficient, using probability theory rather than quantum
mechanics. Because the result can be reached at all, of course it will
be the same as that of quantum mechanics. Otherwise mathematics would
be inconsistent.

\section{The Problem}

There is a transmitter that sends a {\em{signal}} randomly from the
set

\begin{equation}
  S=\{\curvearrowleft\,,\,\curvearrowright\}
\end{equation}

The transmission goes into both of two channels. Each channel attaches
a {\em{tag}} to the signal, according to an algorithm to be specified
below, and re-transmits the tagged signal. The tag comes from the set

\begin{equation}
  T=\{\oplus\,,\,\ominus\}
\end{equation}

The tagging algorithm works as follows. The channel is ``tuned'' by an
angular setting $\zeta\in[0,2\pi]$. Let $r$ represent a number chosen
uniformly from $[0,1]$. Now suppose the signal $\sigma$ is
$\curvearrowleft$. In that case, if $r < \cos^2 \zeta$ then
re-transmit $(\oplus\,,\,\curvearrowleft)$. Otherwise re-transmit
$(\ominus\,,\,\curvearrowleft)$. On the other hand, suppose the signal
is $\sigma$ is $\curvearrowright$. Then, if $r < \sin^2 \zeta$
re-transmit $(\oplus\,,\,\curvearrowright)$, else re-transmit
$(\ominus\,,\,\curvearrowright)$.

At the end of both channels is a receiver-recorder, which makes a
record of received pairs of tagged signals, for some pair of
``tunings'' (or ``$\zeta$-settings'') $(\zeta_1,\zeta_2)$ for the two
channels. For example, one hundred thousand or one million pairs of
tagged signals might be recorded.

Now suppose we map tags to numbers, $T\to T^{\prime}=\{-1,+1\}$, thus:
\begin{align}
  \oplus \mapsto +1 \\
  \ominus \mapsto -1
\end{align}
The problem is to use these numbers to calculate a correlation
coefficient $\rho$, characterizing the correlation between tags in the
received signal pairs, as a function of the difference between the
two~$\zeta$-settings.

\section{Solution}

We will use subscripts to refer to channel numbers. Thus, for example,
$\zeta_2$ may refer to a $\zeta$-setting for channel~2, $\tau_1$ to a
tag value for channel~1, and so on. An unsubscripted letter may refer
to either channel. Thus, for instance, $\tau$ may stand in for either
or both of $\tau_1$ and $\tau_2$. And so on like that.

We will use more or less conventional probability notation, though
also always adding the letter ``$\lambda$'' as a condition, meaning
something such as ``any relevant information we may so far have
neglected.'' It pays to be cautious.\footnote{Bell sometimes
  uses~$\lambda$ to represent what he calls ``residual
  fluctuations''~\cite{bertlmann:manuscript}. Whatever these be, they
  can be considered a portion of what we mean~$\lambda$ to cover.}

The $\zeta$-settings must be set each with respect to its own angular
coordinate system. They cannot be assumed set with respect to the
ether or the distant stars, nor with respect to each other. Therefore
let $\phi_{01},\phi_{02}\in[0,2\pi]$ be the {\em{landmarks}}
(or~{\em{origins}}) for the respective $\zeta$-settings, and
let~$\phi_0$ represent either or both of these landmarks. Let $\phi_1$
and~$\phi_2$ be respective setting values, and let~$\phi$ represent
either or both of these values. Also introduce the notation
\begin{align}
  \Delta\phi_1&=\phi_1-\phi_{01} \\
  \Delta\phi_2&=\phi_2-\phi_{02} \\
  \Delta\phi&=\phi-\phi_{0}
\end{align}
By the problem definition, one immediately gets
\begin{align}
  P(\sigma=\,\curvearrowleft \,|\, \lambda) &= 1\!/2 \\
  P(\sigma=\,\curvearrowright \,|\, \lambda) &= 1\!/2
\end{align}
and
\begin{align}
  P(\tau=\,\oplus \,|\, \sigma=\,\curvearrowleft\,,\; \zeta=\Delta\phi\,,\; \phi_0\in[0,2\pi]\,,\; \lambda) &= \cos^2 \Delta\phi \\
  P(\tau=\,\ominus \,|\, \sigma=\,\curvearrowleft\,,\; \zeta=\Delta\phi\,,\; \phi_0\in[0,2\pi]\,,\; \lambda) &= \sin^2 \Delta\phi \\
  P(\tau=\,\oplus \,|\, \sigma=\,\curvearrowright\,,\; \zeta=\Delta\phi\,,\; \phi_0\in[0,2\pi]\,,\; \lambda) &= \sin^2 \Delta\phi \\
  P(\tau=\,\ominus \,|\, \sigma=\,\curvearrowright\,,\; \zeta=\Delta\phi\,,\; \phi_0\in[0,2\pi]\,,\; \lambda) &= \cos^2 \Delta\phi
\end{align}
Here we are taking a liberty: actually, for~$\phi_{01}$
and~$\phi_{02}$ there should be a joint probability density function
(pdf) specified in the conditions. However, we shall introduce this
pdf only when it is about to be used. For now, consider it implicitly
specified.

We want to construct a table of probabilities of tagged signal pairs
received by the receiver-recorder, so let us start by finding an
expression for the following.
\begin{multline}
  P_1 = P(\sigma=\,\curvearrowleft\,,\; \tau_1=\,\oplus\,,\; \tau_2=\,\oplus \,| \\
  \zeta_1=\Delta\phi_1\,,\; \zeta_2=\Delta\phi_2\,,\; \phi_0\in[0,2\pi]\,,\; \lambda)
\end{multline}
However, because that does not fit well into a line of text, let us
first introduce a shorthand, by writing something like
\begin{equation}
  P_1 = P(\curvearrowleft\,\oplus_1\,\oplus_2\,|\,\phi_1\,,\; \phi_2\,,\;
  [0,2\pi]\,,\; \lambda)
\end{equation}
to mean the same thing. Then, by probability theory, and taking into
account that $\Delta\phi_1$ and~$\Delta\phi_2$ respectively pertain
exclusively to channel~1 or~channel~2 (so conditionality on the
opposite channel's $\Delta\phi$ may be dropped),
\begin{align}
  P_1 &= P(\curvearrowleft\,\oplus_1\,\oplus_2\,|\,\phi_1\,,\; \phi_2\,,\; [0,2\pi]\,,\; \lambda) \\
      &= P(\curvearrowleft\,|\, \lambda) P(\oplus_1\,\oplus_2\,|\,\curvearrowleft\,,\;\phi_1\,,\; \phi_2\,,\; [0,2\pi]\,,\; \lambda) \\
      &= P(\curvearrowleft\,|\, \lambda) P_{11} P_{12} = \frac{1}{2} P_{11} P_{12}
\end{align}
where
\begin{align}
  P_{11} &= P(\oplus_1\,|\,\curvearrowleft\,,\;\phi_1\,,\; [0,2\pi]\,,\; \lambda) = \cos^2 \Delta\phi_1 \\
  P_{12} &= P(\oplus_2\,|\,\curvearrowleft\,,\; \phi_2\,,\; [0,2\pi]\,,\; \lambda) = \cos^2 \Delta\phi_2
\end{align}

By that and similar calculations the following table may be
constructed.
\begin{align}
  P(\curvearrowleft\,\oplus_1\,\oplus_2\,|\,\phi_1\,,\; \phi_2\,,\; [0,2\pi]\,,\; \lambda) &=
  \frac{1}{2} \cos^2 \Delta\phi_1\, \cos^2 \Delta\phi_2 \\
  P(\curvearrowleft\,\oplus_1\,\ominus_2\,|\,\phi_1\,,\; \phi_2\,,\; [0,2\pi]\,,\; \lambda) &=
  \frac{1}{2} \cos^2 \Delta\phi_1\, \sin^2 \Delta\phi_2 \\
  P(\curvearrowleft\,\ominus_1\,\oplus_2\,|\,\phi_1\,,\; \phi_2\,,\; [0,2\pi]\,,\; \lambda) &=
  \frac{1}{2} \sin^2 \Delta\phi_1\, \cos^2 \Delta\phi_2 \\
  P(\curvearrowleft\,\ominus_1\,\ominus_2\,|\,\phi_1\,,\; \phi_2\,,\; [0,2\pi]\,,\; \lambda) &=
  \frac{1}{2} \sin^2 \Delta\phi_1\, \sin^2 \Delta\phi_2 \\
  P(\curvearrowright\,\oplus_1\,\oplus_2\,|\,\phi_1\,,\; \phi_2\,,\; [0,2\pi]\,,\; \lambda) &=
  \frac{1}{2} \sin^2 \Delta\phi_1\, \sin^2 \Delta\phi_2 \\
  P(\curvearrowright\,\oplus_1\,\ominus_2\,|\,\phi_1\,,\; \phi_2\,,\; [0,2\pi]\,,\; \lambda) &=
  \frac{1}{2} \sin^2 \Delta\phi_1\, \cos^2 \Delta\phi_2 \\
  P(\curvearrowright\,\ominus_1\,\oplus_2\,|\,\phi_1\,,\; \phi_2\,,\; [0,2\pi]\,,\; \lambda) &=
  \frac{1}{2} \cos^2 \Delta\phi_1\, \sin^2 \Delta\phi_2 \\
  P(\curvearrowright\,\ominus_1\,\ominus_2\,|\,\phi_1\,,\; \phi_2\,,\; [0,2\pi]\,,\; \lambda) &=
  \frac{1}{2} \cos^2 \Delta\phi_1\, \cos^2 \Delta\phi_2
\end{align}
Disregarding the signal values gives (by addition of probabilities)
\begin{multline}
  \label{coscos}
  P(\oplus_1\,\oplus_2\,|\,\phi_1\,,\; \phi_2\,,\; [0,2\pi]\,,\; \lambda) \\ =
 P(\ominus_1\,\ominus_2\,|\,\phi_1\,,\; \phi_2\,,\; [0,2\pi]\,,\; \lambda) \\ =
   \frac{1}{2} \cos^2 \Delta\phi_1\, \cos^2 \Delta\phi_2 \\
  + \frac{1}{2} \sin^2 \Delta\phi_1\, \sin^2 \Delta\phi_2
\end{multline}
and
\begin{multline}
  \label{cossin}
  P(\oplus_1\,\ominus_2\,|\,\phi_1\,,\; \phi_2\,,\; [0,2\pi]\,,\; \lambda) \\ =
  P(\ominus_1\,\oplus_2\,|\,\phi_1\,,\; \phi_2\,,\; [0,2\pi]\,,\; \lambda) \\ =
  \frac{1}{2} \cos^2 \Delta\phi_1\, \sin^2 \Delta\phi_2 \\
  + \frac{1}{2} \sin^2 \Delta\phi_1\, \cos^2 \Delta\phi_2
\end{multline}

We want to calculate the correlation coefficient
\begin{equation}
  \label{corrcoef}
  \rho = \frac{E(\tau'_1 \tau'_2)}{\sqrt{E({\tau'_1}^2)}\sqrt{E({\tau'_2}^2)}}
\end{equation}
where $\tau'_1, \tau'_2 \in T^{\prime}$ and the expectations $E$ are
calculated with respect to the conditional probabilities derived
above. The numerator is the covariance and the denominator is the
product of the standard deviations.

The choice of values for the elements of $T^{\prime}$ makes it so the
standard deviations in (\ref{corrcoef}) equal one, and thus the
correlation coefficient simplifies to the covariance
\begin{equation}
  \rho = E(\tau'_1 \tau'_2)
\end{equation}
which we now must calculate. To do so, not only must we compute a sum
weighted by the probabilities in (\ref{coscos}) and~(\ref{cossin}),
but we must also eliminate the angular landmarks.

First let us calculate an average of~$\tau'_1 \tau'_2$ weighted by the
probabilities in~(\ref{coscos}) and~(\ref{cossin}), and call that
sum~$\rho^{\prime}$.
\begin{multline}
  \rho^{\prime} 
  = (+1)(+1) P(\oplus_1\,\oplus_2\,|\,\phi_1\,,\; \phi_2\,,\; [0,2\pi]\,,\; \lambda) \\
  + (+1)(-1) P(\oplus_1\,\ominus_2\,|\,\phi_1\,,\; \phi_2\,,\; [0,2\pi]\,,\; \lambda) \\
  + (-1)(+1) P(\ominus_1\,\oplus_2\,|\,\phi_1\,,\; \phi_2\,,\; [0,2\pi]\,,\; \lambda) \\
  + (-1)(-1) P(\ominus_1\,\ominus_2\,|\,\phi_1\,,\; \phi_2\,,\; [0,2\pi]\,,\; \lambda)
\end{multline}
Substituting for the probabilities gives
\begin{multline}
  \rho^{\prime} = \cos^2 \Delta\phi_1\, \cos^2 \Delta\phi_2
  - \cos^2 \Delta\phi_1\, \sin^2 \Delta\phi_2 \\
  - \sin^2 \Delta\phi_1\, \cos^2 \Delta\phi_2
  + \sin^2 \Delta\phi_1\, \sin^2 \Delta\phi_2
\end{multline}
and so
\begin{align}
  \rho^{\prime} &= (\cos^2 \Delta\phi_2 - \sin^2 \Delta\phi_2)
  (\cos^2 \Delta\phi_1 - \sin^2 \Delta\phi_1) \\
  \label{rhoprime}
  &=\cos(2\Delta\phi_2)\,\cos(2\Delta\phi_1)
\end{align}
with the last step by one of the double-angle
identities.\footnote{Though the function does not have the correct
  form and so cannot be the solution, this is probably where a quantum
  physicist would believe they had solved the problem. Their version
  would leave out the landmarks.}

To eliminate the landmarks~$\phi_{01}$ and~$\phi_{02}$, we will have
to compute a double integral weighted by a joint pdf. What pdf to use?
The right side of~(\ref{rhoprime}) is symmetric in~$\phi_{01}$
and~$\phi_{02}$, so let us arbitrarily choose~$\phi_{01}$ to talk
about first. Whatever follows in our arguments must apply to all
possible arrangements, and so the probability density of~$\phi_{01}$
must be uniform, independently of the density
of~$\phi_{02}$. For~$\phi_{02}$, on the other hand, the situation is
different. Our goal is to compute the rotation-invariant correlation
coefficient, given the difference between the two
$\zeta$-settings. Therefore~$\phi_{02}$ must equal~$\phi_{1}$ with
probability one. The necessary joint pdf is thus the Dirac
delta~$\delta(\phi_{02}-\phi_1)$. We do a double integration
of~(\ref{rhoprime}), both integrals going from a point on the
protractor to another.

There will be this complication, however: we have to treat all
landmarks (and thus angular coordinate systems) alike, yet the cosine
sometimes is reversed in sign, which would lead to~$\rho$ being
reversed in sign in some quadrants. Luckily, the plus-or-minus sense
of a correlation coefficient is arbitrary, so we can simply change the
sign back to how we need it. We will also simplify the integrations by
substituting~$\theta=\Delta\phi_1 $. Thus
\begin{multline}
\rho = \int_{0}^{\frac{\pi}{4}}\int_0^{2\pi}\cos(2\Delta\phi_2)\cos(2\theta)\,\delta(\phi_{02}-\phi_1)\,d\phi_{02}\,d\theta \\
+ \int_{\frac{7\pi}{4}}^{2\pi}\int_0^{2\pi}\cos(2\Delta\phi_2)\cos(2\theta)\,\delta(\phi_{02}-\phi_1)\,d\phi_{02}\,d\theta
\end{multline}
if~$\theta\in[0,\frac{\pi}{4}]\cup[\frac{7\pi}{4},2\pi]$,
\begin{equation}
-\rho = \int_{\frac{\pi}{4}}^{\frac{3\pi}{4}}\int_0^{2\pi}\cos(2\Delta\phi_2)\cos(2\theta)\,\delta(\phi_{02}-\phi_1)\,d\phi_{02}\,d\theta
\end{equation}
if~$\theta\in[\frac{\pi}{4},\frac{3\pi}{4}]$,
\begin{equation}
\rho = \int_{\frac{3\pi}{4}}^{\frac{5\pi}{4}}\int_0^{2\pi}\cos(2\Delta\phi_2)\cos(2\theta)\,\delta(\phi_{02}-\phi_1)\,d\phi_{02}\,d\theta
\end{equation}
if~$\theta\in[\frac{3\pi}{4},\frac{5\pi}{4}]$, and
\begin{equation}
-\rho = \int_{\frac{5\pi}{4}}^{\frac{7\pi}{4}}\int_0^{2\pi}\cos(2\Delta\phi_2)\cos(2\theta)\,\delta(\phi_{02}-\phi_1)\,d\phi_{02}\,d\theta
\end{equation}
if~$\theta\in[\frac{5\pi}{4},\frac{7\pi}{4}]$.

In each case the result is the same, so
\begin{equation}
  \rho = \cos(2(\phi_2-\phi_1)) = \cos^2(\phi_2-\phi_1) - \sin^2(\phi_2-\phi_1)
\end{equation}
regardless of quadrant. The formula obviously extends to
all~$\phi_1,\phi_2\in\Re$.

This result accords with quantum mechanics, as it must. Otherwise
either quantum mechanics or our solution would be wrong.

\section{Discussion}

I should not belabor this note with too many details of how Bell went
wrong in his arguments. His writings simply are devoid of
{\em{correct}} methods of logical inference. Instead we encounter
{\em{imitations}} of logical inference.

For example, Bell famously introduces a mathematical contradiction
with the goal of producing a physical
absurdity~\cite{bertlmann:manuscript}. In fact, this is the crux of
his argument in its best known form, but it is unsound reasoning. One
cannot deduce {\em{anything}} from a mathematical contradiction,
except that the mathematical assumptions behind it are wrong. In other
words, all Bell proved is that he made a mistake! Indeed, he made what
is probably the most frequently made mistake in probability theory:
factoring a joint probability incorrectly. However, Bell was
{\em{imitating}} logical inference, not employing the real thing with
skill. For himself and his audience that had not thought this through,
an imitation was good enough.\footnote{Perhaps Bell confused
  mathematical ``{\em{reductio ad absurdum}}''---more properly called
  proof by contradiction---with the more general method of
  argumentation called {\em{reductio ad absurdum}}. But one cannot use
  a mathematical contradiction to prove a general absurdity, because a
  mathematical contradiction is vacuous.}

Perhaps, finally, we have reached the limit of what we should
tolerate. Perhaps, now that millions and millions of dollars have been
wasted, and graduate students' lives increasingly are being wasted, it
is time to put our foot down and say it is enough. Bell was unable to
reason like a true scientist. And now an entire batch of fields
related to his work is supposed to be our society’s greatest
``geniuses,'' but, by a failure in education, its members do not
visualize that a two-channel Bell test is a totally ordinary, causal,
contact-action random process that is, mathematically, shaped like a
pair of wheels, which may have imposed upon them any angular
coordinate system one wishes.

Among the corollaries of that ``shape'' is that, if you rotate
$\phi_1$ and~$\phi_2$ in unison, the frequencies of tag values change
(also in unison)\footnote{Which could be computer-animated in numerous
  interesting ways.} {\em{but the correlation coefficient remains
    invariant}}. Furthermore, it should be experimentally
{\em{impossible}} to violate the so-called ``CHSH
inequality''~\cite{enwiki:1170465048}, if the experiment be properly
designed. This inequality applies a test that itself is not invariant
under rotations of the apparatus. For Bell-test angles, in the case of
photon experiments, a rotation by $\pi\!/8$ should see a drop in
magnitude of the CHSH ``quantum correlation''~$E$ from $\sqrt{1\!/2}$
to~zero, so~$|S|$ (in the ideal) adds up to~$\sqrt2$ instead
of~$2\sqrt2$. The reason is that $E$, in the CHSH formulation, is
{\em{not}} actually the expectation of experimental outcomes, as
claimed. Or, to put it another way, it is an expectation conditional
on a particular coordinate system---but this is not what quantum
mechanics gives, and it is not what one wants. Physicists have left
out the steps that took {\em{our}} calculation from $\rho^{\prime}$
to~$\rho$ proper, and which gave us the {\em{correct}},
{\em{rotation-invariant}} correlation coefficient. Yet experimenters
report positive results. Therefore not only is the inequality itself a
failure of scientific methods, so must be the experiments.

Furthermore, any derivation of the CHSH inequality, like Bell's
reasoning discussed earlier, merely mimics logical reasoning. For
mathematics to be consistent, all methods must reach the same
result. Thus, when the CHSH ``quantum correlation'' is not the same as
the correlation coefficient derived by quantum mechanics, the
conclusion should {\em{not}} be that quantum mechanics is
``different'' from classical physics. The {\em{correct}} conclusion is
that there is an error in the CHSH derivation, and that the researcher
must persist---even if it takes years, and an education in unfamiliar
branches of mathematics---until the derivation is correct. Quantum
physicists, instead, publish a paper ``confirming quantum mechanics,''
then collect citation counts. It is a cushy job, insofar as intellect
is concerned, but this kind of work is what our ``scientific''
community has become. It is a perversion of science.

At this point I shall state unequivocally that there is no such thing
as ``entanglement.'' I write not for the journals but directly to the
reader and so may speak freely. I need not be a cheap philosopher and
hedge with ``There might be action at a distance even though we have
proven conclusively that there need not be.'' No. Oppositely polarized
photons in a Bell test experiment are simply oppositely polarized
photons, not some kind of ``mixture.'' It is obvious that a
``superposition state'' or any such verbally obfuscated stuff must
always simply be different notation for some probability expression,
wave coherence expression, or the like. The real reason anyone
believes there is magic involved seems to be not that the magic
exists, but that some Fathers of the early Quantum Church (although
perhaps not Niels~Bohr himself~\cite{enwiki:1174555777}) strictly
forbade treatment of quantum mechanics in the classical
fashion. Physicists have been true to that stricture since and simply
do not {\em{permit}} the heresy of a solution by classical
methods.\footnote{Indeed, some Bell tests are soluble {\em{by
      electromagnetic wave theory}}~\cite{kracklauer:nonloc-or-geom},
  which is as classical a physics as one can imagine, but this fact is
  dismissed one way or another by the orthodoxy.} Our universities are
cult indoctrination centers.\footnote{As an engineering student in the
  early 1980s, the author himself encountered what apparently was a
  common practice in approximately the 1970s: the professor would
  {\em{boom}}: ``{\em{Do {\em{not}} try to understand quantum
      mechanics! Just do the calculations!}}'' This was an actual
  {\em{command}}. Being {\em{commanded}} by my physics professor shook
  me painfully and permanently---particularly because, in general, he
  was the best teacher I ever had, who gave instructive homework and
  never made the exams cruelly difficult.} One is permitted
{\em{only}} to reach some fake ``solution'' different from that of
quantum mechanics, which then is used to ``prove'' that quantum
mechanics is ``different.''  Otherwise physicists would have
discovered long ago what we discovered by example above: that there
actually {\em{is no}} ``modern physics''---just ``classical physics''
that no one has been allowed to investigate.

Physicists, in their roles as researchers and professors, apparently
do not teach themselves and their students {\em{how}} to reach
solutions of random process problems, except by the ``magic'' of
quantum mechanics. {\em{I}} knew how because {\em{I}} majored not in
physics at all, but in the very closely related subject of
{\em{electrical engineering}}---where mistakes are punishable by
corporate bankruptcy, and sometimes even by prison for the
engineer. In basic research there may be no consequences for simply
repeating orthodoxy, be it right or wrong. Peer review further
reinforces the orthodoxy. Orthodox papers in quantum physics are
reviewed by orthodox quantum physicists for publication in orthodox
journals. Peer review is useless, in this instance, to serve the
scientific mission. The papers never are subjected to analysis by
experts in random process analysis, who {\em{are not}} quantum
physicists, and who would tear the arguments to shreds. The papers are
reviewed instead by persons ignorant of random process analysis, who
delight in conclusions that reinforce orthodox doctrine. Careers rise
as the peers see each others’ papers published in the most elite of
journals. Eventually, the Nobel Prize in Physics for~2022 is awarded
for superlative achievements in the pursuit of what, in fact, is
balderdash~\cite{enwiki:1169021780}. But the winners cannot see that
it is balderdash. Few in the field have even the background to see
that the achievement was nothing at all.

The author of {\em{this}} paper would tell a Nobel Committee to take
their prize and dump it in {\em{Östersjön}}, and recommends the same
for all Nobels in science and mathematics. He does not believe in
prizes and awards, and thinks that no scientist should be labeled as
privileged over another. An artificial reef of Nobel medals could form
a home for fish and crabs. Prizes for science are unethical and
destructive at best~\cite{kohn1993punished,sciencemag:nobelbad}, but
now also shown to be awarded without distinction to both actual
discovery and the pursuit of nothing. Which is better, to be insulted
with a cash reward for success in a system that is a perversion of
science---which rewards publication and citation-counting over
self-education and stubborn persistence---or to be a humble pursuer of
the {\em{actual}} scientific enterprise, which is to clear up
mysteries~\cite{clearingupmysteries}? Let the gold medals belong to
the crabs.

\bibliography{eprb_signal_correlations}{}
\bibliographystyle{IEEEtran}

\end{document}
