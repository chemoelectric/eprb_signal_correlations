%% THIS TECHNOTE IS A WORK IN PROGRESS

\documentclass[9pt,technote]{IEEEtran}
\usepackage{amsmath,amssymb,amsfonts}
\usepackage{algorithmic}
\usepackage{algorithm}
\usepackage{array}
\usepackage[caption=false,font=normalsize,labelfont=sf,textfont=sf]{subfig}
\usepackage{textcomp}
\usepackage{stfloats}
\usepackage{url}
\usepackage{verbatim}
\usepackage{graphicx}
\usepackage{cite}
\hyphenation{op-tical net-works semi-conduc-tor IEEE-Xplore}

\begin{document}

\title{Correlation Coefficient of a Two-Channel Bell Test\\by Probability Theory}
\author{Barry~Schwartz,~\IEEEmembership{Member,~IEEE}}

\maketitle

\begin{abstract}
  It is widely believed that only quantum mechanics can characterize
  the correlations of a Bell test. In fact, quantum mechanics is
  superfluous to the problem, and a correlation coefficient may be
  derived by probability theory.
\end{abstract}

\section{Introduction}

John~S. Bell famously proposed a test {\bf{FILL IN REFERENCE HERE}}
supposed to prove that quantum mechanics was, in essence, magical. I
put the matter this way because the proposed test is soluble by
probability theory, and yet Bell seems to have claimed quantum
mechanics alone could provide the One True Answer that surpassed all
other solutions. At the time of this writing, Bell's test is
considered passed and the matter resolved in favor of the One True
Answer.

Of course, what actually happened is that Bell {\em{did not know}} the
test could be solved by any means other than quantum mechanics, and
surely {\em{already believed}} that quantum mechanics was
magical. Quantum mechanics has widely been considered magical for a
very long time. However, the Bell test problem is not, in fact, one of
quantum mechanics at all. It is more properly a problem in random
signal analysis. It merely happens that quantum mechanics ``knows''
how to solve such a problem, without the user having to understand
signal analysis.

We, however, wish to understand---and so shall recast the problem as
one of signal analysis, and shall solve it.

\section{The Problem}

There is a transmitter that sends a {\em{signal}} randomly from the
set

\begin{equation}
  S=\{\curvearrowleft\,,\,\curvearrowright\}
\end{equation}

The transmission goes into both of two channels. Each channel attaches
a {\em{tag}} to the signal, according to an algorithm to be specified
below, and re-transmits the tagged signal. The tag comes from the set

\begin{equation}
  T=\{\oplus\,,\,\ominus\}
\end{equation}

The tagging algorithm works as follows. The channel is ``tuned'' by an
angular setting $\zeta\in[0,2\pi]$. Let $r$ represent a number chosen
uniformly from $[0,1]$. Now suppose the signal $\sigma$ is
$\curvearrowleft$. In that case, if $r < \cos^2 \zeta$ then
re-transmit $(\oplus\,,\,\curvearrowleft)$. Otherwise re-transmit
$(\ominus\,,\,\curvearrowleft)$. On the other hand, suppose the signal
is $\sigma$ is $\curvearrowright$. Then, if $r < \sin^2 \zeta$
re-transmit $(\oplus\,,\,\curvearrowright)$, else re-transmit
$(\ominus\,,\,\curvearrowright)$.

At the end of both channels is a receiver-recorder, which makes a
record of received pairs of tagged signals, for some pair of
``tunings'' $(\zeta_1,\zeta_2)$ for the two channels. For example, one
hundred thousand or one million pairs of tagged signals might be
recorded.

Now suppose we map tags to numbers, $T\to T'=\{-1,+1\}$, thus:
\begin{align}
  \oplus \mapsto +1 \\
  \ominus \mapsto -1
\end{align}
The problem is to use these numbers to calculate a correlation
coefficient $\rho$, characterizing the correlation between tags in the
received signal pairs.

\section{Solution}

We will use subscripts to refer to channel numbers. Thus, for example,
$\zeta_2$ may refer to a $\zeta$ parameter for channel~2, $\tau_1$ to
a tag value for channel~1, and so on. A plain letter without a
subscript may refer to either channel. Thus, for instance, $\tau$ may
stand in for either $\tau_1$ or $\tau_2$. And so on like that.

We will use more or less conventional probability notation, though
also always adding the letter ``$\lambda$'' as a condition, meaning
something such as ``any relevant information we may so far have
neglected''. It pays to be cautious.

Note also that all angles must be with respect to some origin, which
we will call $\phi_1$ and associate, purely for convenience, with the
``tuning'' of channel~1. Thus ``$\phi_1\in[0,2\pi]$'' will appear
often as a condition. The correlation coefficient we calculate,
however, will be coordinate-free---as it must be, because the angular
origin is arbitrary. The angular origin will appear in the final
expression only as a term in a difference.

By the problem definition, one immediately gets
\begin{align}
  P(\sigma=\,\curvearrowleft \,|\, \lambda) &= \frac{1}{2} \\
  P(\sigma=\,\curvearrowright \,|\, \lambda) &= \frac{1}{2}
\end{align}
and
\begin{align}
  P(\tau=\,\oplus \,|\, \sigma=\,\curvearrowleft\,,\; \zeta=\phi\,,\; \phi_1\in[0,2\pi]\,,\; \lambda) &= \cos^2 \phi \\
  P(\tau=\,\ominus \,|\, \sigma=\,\curvearrowleft\,,\; \zeta=\phi\,,\; \phi_1\in[0,2\pi]\,,\; \lambda) &= \sin^2 \phi \\
  P(\tau=\,\oplus \,|\, \sigma=\,\curvearrowright\,,\; \zeta=\phi\,,\; \phi_1\in[0,2\pi]\,,\; \lambda) &= \sin^2 \phi \\
  P(\tau=\,\ominus \,|\, \sigma=\,\curvearrowright\,,\; \zeta=\phi\,,\; \phi_1\in[0,2\pi]\,,\; \lambda) &= \cos^2 \phi
\end{align}


\end{document}
