%% THIS TECHNOTE IS A WORK IN PROGRESS

\documentclass[9pt,technote]{IEEEtran}
\usepackage{amsmath,amssymb,amsfonts}
\usepackage{algorithmic}
\usepackage{algorithm}
\usepackage{array}
\usepackage[caption=false,font=normalsize,labelfont=sf,textfont=sf]{subfig}
\usepackage{textcomp}
\usepackage{stfloats}
\usepackage{url}
\usepackage{verbatim}
\usepackage{graphicx}
\usepackage{cite}
\hyphenation{op-tical net-works semi-conduc-tor IEEE-Xplore}

\newcommand{\signal}{{\em{signal}}}
\newcommand{\tagg}{{\em{tag}}}

\begin{document}

\title{Solution to a Two-Channel Bell Test\\by Probability Theory}
\author{Barry~Schwartz,~\IEEEmembership{Member,~IEEE}}

\maketitle

\section{Introduction}

John~Bell famously proposed a test {\bf{FILL IN REFERENCE HERE}}
supposed to prove that quantum mechanics was, in essence, magical. I
put the matter this way because the proposed test is soluble by
probability theory, and yet Bell seems to have claimed quantum
mechanics alone could provide the One True Answer that surpassed all
other solutions.

Of course, what actually happened is that Bell {\em{did not know}} the
test could be solved by any means other than quantum
mechanics. However, the problem is not, in fact, one of quantum
mechanics at all. It is more properly a problem in random signal
analysis. It merely happens that quantum mechanics ``knows'' how to
solve such a problem, without the user having to understand signal
analysis.

We, however, wish to understand---and so shall recast the problem as
one of signal analysis, and shall solve it.

\section{The Problem}

There is a transmitter that sends a \signal\ randomly from the set

$$S=\{\curvearrowleft,\curvearrowright\}$$

The transmission goes into both of two channels. Each channel attaches
a \tagg\ to the signal, according to an algorithm to be specified
below. The \tagg\ comes from the set

$$T=\{\oplus,\ominus\}$$

\end{document}
